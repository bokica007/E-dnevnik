\documentclass[14pt,a4paper]{book}
\usepackage[T1]{fontenc}
\usepackage[utf8]{inputenc}

\usepackage{fancyhdr}
 
\pagestyle{fancy}

\title{\textbf{Informacioni sistem : Elektronski dnevnik}}
\author{Tim : Miroslav Maksimovic, Dobrinko Drinic, Nevena Vasic}
\date{Oktobar 15, 2016 }


\begin{document}
  \maketitle
\section*{Cilj informacionog sistema e-Dnevnik }
 Cilj projekta e-Dnevnik je informatizovati proces sluzbene evidencije o razrednim odjeljenjima u skoli i olaksati i prosiriti upotrebu i koristenje razredne knjige.\newline 
 Dok u klasicnom dnevniku profesor upisuje casove i odsutne ucenike rucno, u elektronskom dnevniku postoje predefinisane stavke koje profesor potvrđuje jednim klikom.

\section*{Zasto se pravi informacioni sistem e-Dnevnik ?  }
Ideja e-Dnevnika-a je da zainteresovanim stranama (roditelj,skola,razredni,nastavnik) pruzi pravovremenu informaciju sta se desava sa decom u skoli. \newline

Za razliku od tradicionalnog pristupa administraciji, elektronski pristup je veoma brži, pa profesori mogu više vremena da posvete radu sa učenicima i osmišljavanju kreativne nastave. \newline

Osnovna funkcija elektronskog dnevnika jeste da posreduje između nastavnika i učenika, ali i između nastavnika i roditelja. \newline

 Digitalizacijom klasičnog školskog dnevnika koji se obično koristi u školama,dolazi do nekih inovacija koje nisu postojale ranije. \newline
 
Zahvaljujući upotrebi elektronskog dnevnika u nastavi, nema neobjektivnosti u odnosu sa učenicima (neobjektivno ocenjivanje, pravdanje izostanaka…).


\section*{Koje zadatke informacioni sistem e-Dnevnik mora da izvrsava ?} 
\enumerate
\item -	Elektronska arhiva ocjena
\item -	Elektronski unos ocena 
\item -	Elektronski unos izostanaka
\item -	Elektronsko zakljucivanje ocena
\item -	Email obavestenja o stanju ucenika za roditelje (moze da odredi razrednik 1sedmicno,1 mjeceno) 
\item -	Stampa svjedocanstva iz elektronskog dnevnika
\item -	Statistika za razredne starjesine
\item -	Statistika za odeljenska veca 
\item -	Statistika  po predmetima
\item -	Datum za roditeljske sastanke 
 \newpage
Ko ima pristup e-dnevniku?
 \enumerate
\item -Direktori skola i radnici u pedagoskim sluzbama imaju pun pristup elektronskom dnevniku
\item -Predmetni nastavnici imaju pristup samo podacima o svom predmetu
\item - Razredne starjesine imaju pristup svim podacima o uceniku svoga razreda
\item -Roditelji imaju pristup svim podacima o svojoj djeci (Razmisliti) 

\section*{Od cega se sastoji informacioni sistem e-Dnevnik ?}
E-Dnevnik se sastoji  od baze podataka ucenika i grafickog korisnickog interfejsa koji varira u odnosu na to koje privilegije ima dati korisnik ( Direktor,predmetni nastavnik,razredni starjesina, roditelji )

\section*{Kako se implementira informacioni sistem e-Dnevnik  ?}
Implemetira se pomocu sledecih tehnologija :
\newline
\enumerate
\item[-]  MySql (http://www.mysql.com- Structured Query Language) – baza podataka 
\item[-] -	Microsoft Visual studio 2015.2  ( https://www.visualstudio.com )

\end{document}